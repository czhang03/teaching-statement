%%%%%%%%%%%%%%%%%%%%%%%%%%%%%%%%%%%%%%%%%
% "ModernCV" CV and Cover Letter
% LaTeX Template
% Version 1.1 (9/12/12)
%
% This template has been downloaded from:
% http://www.LaTeXTemplates.com
%
% Original author:
% Xavier Danaux (xdanaux@gmail.com)
%
% License:
% CC BY-NC-SA 3.0 (http://creativecommons.org/licenses/by-nc-sa/3.0/)
%
% Important note:
% This template requires the moderncv.cls and .sty files to be in the same
% directory as this .tex file. These files provide the resume style and themes
% used for structuring the document.
%
%%%%%%%%%%%%%%%%%%%%%%%%%%%%%%%%%%%%%%%%%

%----------------------------------------------------------------------------------------
%	PACKAGES AND OTHER DOCUMENT CONFIGURATIONS
%----------------------------------------------------------------------------------------

\documentclass[11pt,a4paper,sans]{moderncv} % Font sizes: 10, 11, or 12; paper sizes: a4paper, letterpaper, a5paper, legalpaper, executivepaper or landscape; font families: sans or roman

\moderncvstyle{classic} % CV theme - options include: 'casual' (default), 'classic', 'oldstyle' and 'banking'
\moderncvcolor{orange} % CV color - options include: 'blue' (default), 'orange', 'green', 'red', 'purple', 'grey' and 'black'

\usepackage{biblatex} %Imports biblatex package
\addbibresource{refs.bib} %Import the bibliography file

\makeatletter
\NewDocumentCommand{\mysubsection}{sm}{%
  \par\addvspace{1ex}%
  \phantomsection{}% reset the anchor for hyperrefs
  \addcontentsline{toc}{subsection}{#2}%
  {\strut\raggedleft\raisebox{\baseletterheight}{\color{color1}\rule{0.3\hintscolumnwidth}{0.95ex}}\quad}{\strut\subsectionstyle{#2}}%
  \par\nobreak\addvspace{.5ex}\@afterheading}% to avoid a pagebreak after the heading
\makeatother


\usepackage[margin=1.5cm]{geometry} % Reduce document margins
%\setlength{\hintscolumnwidth}{3cm} % Uncomment to change the width of the dates column
%\setlength{\makecvtitlenamewidth}{10cm} % For the 'classic' style, uncomment to adjust the width of the space allocated to your name

%----------------------------------------------------------------------------------------
%	NAME AND CONTACT INFORMATION SECTION
%----------------------------------------------------------------------------------------

\firstname{Cheng} % Your first name
\familyname{Zhang} % Your last name

% All information in this block is optional, comment out any lines you don't need
\title{Teaching Statement}
%\address{W. Ethan Eagle}{}
%\mobile{(302) 584 3464}
%\phone{(000) 111 1112}
%\fax{(000) 111 1113}
\email{czhang03@bu.edu}                               % optional, remove / comment the line if not wanted
\homepage{czhang03.github.io}              % optional, remove / comment the line if not wanted
\social[github]{czhang03}                              % optional, remove / comment the line if not wanted
\extrainfo{\faFile{}~\href{https://media.githubusercontent.com/media/czhang03/CV/master/CV.pdf}{Curriculum vitae}}

%\homepage{https://czhang03.github.io/}{https://czhang03.github.io/} % The first argument is the url for the clickable link, the second argument is the url displayed in the template - this allows special characters to be displayed such as the tilde in this example
%\extrainfo{additional information}
%\photo[70pt][0.4pt]{pictures/picture} % The first bracket is the picture height, the second is the thickness of the frame around the picture (0pt for no frame)

% reset quote width
\let\originalrecomputecvlengths\recomputecvlengths
\renewcommand*{\recomputecvlengths}{%
\originalrecomputecvlengths%
\setlength{\quotewidth}{0.75\textwidth}}
\quote{``Tell me and I forget, teach me and I may remember, involve me and I learn.'' -- Benjamin Franklin }

%----------------------------------------------------------------------------------------

\begin{document}
\makecvtitle % Print the CV title

% set spacing
\setlength\parskip{8px}
% avoid paraskip on the first paragraph
\vspace{-\parskip} 

My undergraduate experience at a liberal arts college was transformative, thanks to the exceptional teaching and mentorship I received from my professors. 
Their guidance and wisdom not only shaped my academic pursuits but also had a profound impact on my personal growth and development. 
I have been fortunate to maintain lasting relationships with many of my mentors, who continue to inspire me with their counsel. 
Through my teaching, I aspire to emulate their success by making a lasting difference in the lives of my students. 
My goal is to provide principled and transferable skills that will empower my students throughout their life, regardless of the paths they choose. 
By engaging and inspiring my students, I aim to equip them with the skills, confidence, and critical thinking abilities necessary to succeed in our rapidly-changing world.

\section{Teaching Philosophy And Interests}

The wide availability of information has transformed the way we learn and works. 
Rather than viewing emerging technologies as a threat to education, I see them as an opportunity to shift teaching focus from mundane memorization to mastering abstract core concepts, applying them to realistic problems, and inspiring the love of learning.
While I don't pretend to have all the answers to the challenges facing education today, my experiences as both a student and a teacher have led me to identify several key strategies that can help achieve this goal.
\begin{itemize}
  \item \textbf{Core Concepts:}
  Simplicity is the key to engaging and effective learning, thus my classes are always centered around several widely-applicable yet simple core concepts. 
  While students may not remember every minor detail, they should be able to recall and apply these fundamental ideas for problem-solving.
  \item \textbf{Grounded Presentations:} 
  Abstraction is essential, but it must be grounded in context to be meaningful. 
  To provide such contexts, I use a combination of real-world examples, interesting connections, and hands-on practice to help students learn abstract concepts. 
  I like to introduce abstract ideas through examples in class, then provide opportunities for students to practice and reinforce their understanding through homework, labs, and exams.
  \item \textbf{Engagement:}
  I have observed that students often resort to cheating when they feel overwhelmed by the material or anxious about their grades. 
  Rather than relying strictly on cheat-detection systems, I focus on engaging these students and addressing the underlying issues. 
  Specifically, I design the homework and exams to provide constructive feedback and support, rather than discouragement. 
  With inspiration from game design~\cite{engelstein_AchievementRelockedLoss_2020,lewis_MotivationalGameDesign_2012}, I identified two principles to achieve this goal: allow students to quickly recover from mistakes, and have a clear alternative path to success.
  \item \textbf{Exploration:} Besides homework, exam, and extra credit problems, I strive to also provide abundant opportunities to mentor student to further explore their interest. 
  These include study/reading groups, research seminars, and undergrad research mentoring.
\end{itemize}

\textbf{Teaching Interest:}
I have a passion for teaching a wide variety of classes, as demonstrated by my diverse teaching experience, including programming language, and functional programming, linear algebra, abstract algebra, and basic number theory. 
However, my academic background in Kleene algebra has equipped me with specialized knowledge, which facilitates teaching topics like functional programming, programming languages, automata theory, and algebra. 
In general, I take great pleasure in breaking down complex topics into manageable components and linking them around a few key concepts; this approach helps students maintain a clear mental image of the class materials and grasp these contents more effectively. 
My greatest joy comes from witnessing students not only succeed but also innovate by applying these foundational ideas to tackle complex real-world problems, and uncover connections on their own.
Ultimately, I aspire to impart not only specific knowledge but also the underlying ideas, thereby inspiring students' passion for the subject and their love for learning in general.

\section{Case Study: Functional Programming Class With Haskell}

To illustrate my teaching philosophy, I'd like to share my experience co-designing a Haskell course, "CS320 Concepts of Programming Language" in Spring 2019.
I will walk through my thought process and highlight the key improvements I aim to incorporate into my future teaching practices.

We chose Haskell as the teaching language due to its clear and consistent syntax and semantics. 
In contrast to impure languages like Ocaml or Scala, Haskell's pure functional programming paradigm forces students to think functionally and avoid relying on impure constructs like references and loops.

To support students, I maintained an extensive online presence through Piazza and held sufficient office hours to help students with challenging questions. 
Notably, I intentionally scheduled office hours away from homework deadlines to encourage students to start assignments early and engage in collaborative discussions on Piazza after office hours.
We also implemented a generous grace period for submissions, averaging around half a day, and allowed resubmissions for up to two assignments at the end of the semester. 

To provide rapid feedback for homework, our grading system utilized QuickCheck which provides immediate counter-examples to facilitate effective debugging. 
While this approach required significant effort to ensure that our tests were thorough, yet did not reveal answers, I believe that the benefits of immediate offline feedback outweighed the costs. 
However, I acknowledge that mere counterexamples may not be sufficient for students with fundamental conceptual misunderstandings. 
To address this, I encouraged students to seek help online or in study groups. 
In the future, I am considering the development of misconception-based teaching tools, such as the LTL Tutor~\cite{plt_LTLTutor_2024}, to further support student learning. 
I am always eager to collaborate with social scientists, HCI experts, and students to design effective teaching systems that cater to diverse learning needs.

Finally, to develop the class material, I first identified four core concepts: functional programming, algebraic data types, type-driven development, and monads. 
I recognized that monads would be the most challenging concept to teach, so I decided to introduce them as a design pattern instead of as a mathematical concept, while providing abundant examples to make the concept more accessible to students.

The curriculum was then designed around these four core concepts, starting with an introduction to Haskell without prelude. 
I guided students through the implementation of familiar structures like boolean values, natural numbers, lists, and maybe types, allowing them to practice defining, destructing, and recursing on algebraic data types. 
Next, I had students implement and use functional primitives like ``map'', ``filter'', and ``fold'' to familiarize them with manipulating collection data structures.
When implementing these primitives, I demonstrated type driven development by listing the type of all the available arguments, and showed that functions like ``map'' and ``fold'' can be implemented in one meaningful way.

I then introduced the data type of reader and state monad, without introducing monads, and had students work on small computations using these types. 
Once students were comfortable tracing the monadic computations, I introduced the concept of monads and do expressions. 
The final project, which was split into several assignments, asked students to implement a simple parser using the parser monad. 
Most students were able to successfully implement the parser monad using type-driven development, understand the semantics of the monad, and create a functional parser. 
Many were even able to handle associativity with the help of examples.

I believe that this class design reflects my teaching philosophy, which focuses on a few important abstract concepts while providing students with rapid feedback and support. 
As a result, students felt a sense of progress and accomplishment as they solved problems, but not punished when they fail. 
The success of our approach was evident in the final project, where most students were able to produce a competent parser in just a few weeks, with no prior experience, demonstrating effective utilization of the four core concepts.

\section{Teaching and Organizing Experiences}

Throughout my academic journey at Boston University, I had extensive experience in event organization and teaching, building a track record of engaging learning experiences for both graduate and undergraduate students.

As the founder and organizer of several study groups and seminars, I have successfully fostered a culture of collaborative learning and intellectual exploration among students of all levels. 
Notably, I organized the long-running BU POPV seminar and reading groups focused on Coq/SSReflect and category theory, which attracted researchers and students from within and beyond the university to learn and share innovative research ideas.

As a teaching assistant for a wide range of courses, including introductory programming, algebraic algorithms (basic number theory and abstract algebra), geometric algorithms (linear algebra), and programming languages, I honed my skills in classroom management, student support, and curriculum design. 
Serving as a TA in many classes has not only shaped my teaching philosophies, but also given me invaluable experience in leading grading teams and developing course materials.

In addition to my teaching assistant roles, I have had the privilege of mentoring two talented undergraduate researchers through the UR2PhD program. 
Over the course of a year, we worked together on cutting-edge research in Kleene Algebra, culminating in a paper submission to LICS 2025, one of the most prestigious conferences in theoretical computer science.

\textbf{Course Evaluations:}
During my last year of teaching, I have consistently been rated above 4/5 on average, and have received overwhelmingly positive feedback, as illustrated by the following selected evaluations:

\emph{``Cheng often does this thing where he creates a map and connects all of the concepts in Linear Algebra together which is the best thing ever[,] because so much of this class builds on top of previous material, so it really helps you understand concepts being taught.''}\\[3px]
\emph{``Cheng Zhang is the most helpful instructor I've ever seen. He responds super fast in Piazza while being able to explain complicated concepts clearly. Also, he is patient and friendly. I rate him a 100 out of 10. Love you :)''}\\[3px]
\emph{``Cheng was a great TF. He wanted us students to learn with a passion. I wish there were more TF's like Mark and Cheng.''}\\[3px]
\emph{``Helpful in office hours and willing to schedule time outside of those in order to help.''}

\newpage
\printbibliography %Prints bibliography


\end{document}
